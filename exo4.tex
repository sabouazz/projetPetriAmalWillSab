\section*{Exercice 4~: Spécification d'une construction d'une maison}
\subsection*{Question 1}

TODO (below : just an example in tikz)

\begin{tikzpicture}[node distance=1.3cm,>=stealth',bend angle=45,auto]

  \tikzstyle{place}=[circle,thick,draw=blue!75,fill=blue!20,minimum size=6mm]
  \tikzstyle{red place}=[place,draw=red!75,fill=red!20]
  \tikzstyle{transition}=[rectangle,thick,draw=black!75,
  			  fill=black!20,minimum size=4mm]

  \tikzstyle{every label}=[red]

  \begin{scope}
    % First net
    \node [place,tokens=1] (rac-u)                           {};
    \node [place] (c1) [below of=rac-u]                      {};
    \node [place] (s)  [below of=c1,label=above:$s\le 3$] {};
    \node [place] (c2) [below of=s]                       {};
    \node [place,tokens=1] (repos-c) [below of=c2]                      {};

    \node [transition] (e1) [left of=c1] {}
      edge [pre,bend left]                  (rac-u)
      edge [post,bend right]                (s)
      edge [post]                           (c1);

    \node [transition] (e2) [left of=c2] {}
      edge [pre,bend right]                 (repos-c)
      edge [post,bend left]                 (s)
      edge [post]                           (c2);

    \node [transition] (l1) [right of=c1] {}
      edge [pre]                            (c1)
      edge [pre,bend left]                  (s)
      edge [post,bend right] node[swap] {4} (rac-u);

    \node [transition] (l2) [right of=c2] {}
      edge [pre]                            (c2)
      edge [pre,bend right]                 (s)
      edge [post,bend left]  node {2}       (repos-c);
  \end{scope}

  \begin{pgfonlayer}{background}
    \filldraw [line width=4mm,join=round,black!10]
      (rac-u.north  -| l1.east)  rectangle (repos-c.south  -| e1.west);
  \end{pgfonlayer}
\end{tikzpicture}

\subsection*{Question 2}
\subsection*{Question 3}

Bien que les deux réseaux de pétri conservent la sémantique, le
deuxième semble plus naturel dans la mesure où l'on n'introduit pas
d'état artificiel d'entre-action. Modéliser les tâches par des places
convient donc mieux, les transitions servant à exprimer le passage
d'une tâche à l'autre, notamment la complétion d'une tâche.
