\documentclass[a4paper, 12pt]{article}
%\usepackage[utf8]{inputenc} 
\usepackage[frenchb]{babel}
\usepackage{fullpage}
\usepackage[T1]{fontenc} 
\usepackage{graphicx}  
\usepackage[final]{pdfpages}
\usepackage{amsmath, amssymb,amsthm}
\usepackage{algorithm,algorithmic}
\usepackage{listingsutf8}
\usepackage{lmodern}
\usepackage[tikz]{bclogo}
\usepackage{pgfplots}
\usepackage{fancybox}
\usepackage{makecell}
\usepackage{array, multirow, tabularx}
\usepackage{xcolor}
\usepackage{framed, color}
\setlength{\parskip}{\bigskipamount}


\newtheorem{mydef}{Définition}
\newtheorem{thm}{Théorème}
\newtheorem{lem}{Lemme}
\newtheorem{cor}{Corollaire}
\newtheorem{prop}{Propriété}
\newtheorem{conj}{Conjecture}

\title{TP M2: Réseau de Pétri} \author{William Dyce, Amal Mejdoub, Sabrina Ouazzani-Chahdi} 

\date{Décembre 2012}

\begin{document}

\newcommand\bcdef{\includegraphics[width=0.6cm]{def.png}}

\maketitle

\begin{bclogo}[couleur=red!30, arrondi=0.1, logo= \bcdef,
    ombre=true, epOmbre=0.25, couleurOmbre=black!30, epBarre=1,
    barre=zigzag]{Blabla} hop !
\end{bclogo}

\section{Exercice 1}
\subsection{Question 1}
à faire sous tina
\subsection{Question 2}
En $p_1$, le pochoir ne reçoit aucun signal.
En $t_1$
Notons $A_d$ et $R_d$ respectivement avancer et reculer le pochoir de
droite.
Notons $A_g$ et $R_g$ respectivement avancer et reculer le pochoir de gauche.
Quand un pochoir avance, il ne recule pas.
On commence donc pas le faire avancer ($p_3$ et $p_2$) puis reculer ($p_4$ et $p_5$)
On distingue un bloc parallélisme dans la figure 3 qui signie que les
pochoirs droite et gauche sont actionnés en parallèle.

\subsection{Question 3}

\subsection{Question 3}

\end{document}
