\section*{Exercice 7~: Sémaphore}
\subsection*{Question 1}
\begin{center}
\begin{tikzpicture}[node distance=1.3cm,>=stealth',bend angle=45,auto]

  \tikzstyle{place}=[circle,thick,draw=blue!75,fill=blue!20,minimum size=6mm]
  \tikzstyle{red place}=[place,draw=red!75,fill=red!20]
  \tikzstyle{transition}=[rectangle,thick,draw=black!75,
  			  fill=black!20,minimum size=4mm]

  \tikzstyle{every label}=[red]

  \begin{scope}
    \node [place] (w1)      [label=above:$exec1$] {};
    \node [place] (w2)      [below of=w1, label=above:$exec2$] {};
    \node [place,tokens=1] (s)  [below of=c1] {};

    \node [transition] (e1) [below left of=w2,label=right:$P2$] {}
      edge [pre,bend left]                  (w2)
      edge [post,bend right]                (s);

    \node [transition] (e2) [left of=e1,label=right:$P1$] {}
      edge [pre,bend left]                  (w1)
      edge [post,bend right]                (s);

   \node [transition] (l1) [below right of=w2,label=left:$V2$] {}
      edge [pre,bend left]                  (s)
      edge [post,bend right] node[swap] {} (w2);
      
   \node [transition] (l2) [right of=l1,label=left:$V1$] {}
      edge [pre,bend left]                  (s)
      edge [post,bend right] node[swap] {} (w1);
  \end{scope}
\end{tikzpicture}
\end{center}

Par définition d'un sémaphore, $P$ et $V$ sont totalement
parallèlisés, et ne sont donc pas en concurrence l'un par rapport à
l'autre. Cependant, ils peuvent être en concurrence par rapport à
eux-même.

\subsection*{Question 2}

On représente une couleur par processus.
 
\begin{center}
\begin{tikzpicture}[node distance=1.3cm,>=stealth',bend angle=45,auto]

  \tikzstyle{place}=[circle,thick,draw=blue!75,fill=blue!20,minimum size=6mm]
  \tikzstyle{red place}=[place,draw=red!75,fill=red!20]
%\tikzstyle{token}=[token,draw=red!75,fill=red!20]
  \tikzstyle{transition}=[rectangle,thick,draw=black!75,
  			  fill=black!20,minimum size=4mm]

  \tikzstyle{every label}=[red] 
\tikzstyle{token}=
            [fill=red,draw=none,circle, inner sep=0.5pt,minimum
              size=1ex, text=white]

  \begin{scope}
    \node [place] (w1)      [label=above:$exec$] {};
    \node [place, tokens=1] (s)  [below of=w1] {};

    \node [transition] (e2) [below left of=w1,label=left:$P$] {}
      edge [pre,bend left] node {<R,V,B>}    (w1)
     edge [post,bend right]      node[swap] {<R,V,B>}          (s);
      
   \node [transition] (l2) [below right of=w1,label=right:$V$] {}
      edge [pre,bend left]         node {<R,V,B>}         (s)
      edge [post,bend right] node[swap] {<R,V,B>} (w1);
  \end{scope}
\end{tikzpicture}
\end{center}


Dans la version colorée du réseau de Pétri du sémaphore (par exemple,
les couleurs RVB proposées ici), $P$ et $V$ demeurent non concurrents
l'un par rapport à l'autre. Grâce aux couleurs associées aux
processus, ils ne sont également pas concurrents par rapport à eux
même.
