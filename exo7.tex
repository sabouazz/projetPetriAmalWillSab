\section*{Exercice 7~: Sémaphore}
\subsection*{Question 1}
\begin{center}
\begin{tikzpicture}[node distance=1.3cm,>=stealth',bend angle=45,auto]

  \tikzstyle{place}=[circle,thick,draw=blue!75,fill=blue!20,minimum size=6mm]
  \tikzstyle{red place}=[place,draw=red!75,fill=red!20]
  \tikzstyle{transition}=[rectangle,thick,draw=black!75,
  			  fill=black!20,minimum size=4mm]

  \tikzstyle{every label}=[red]

  \begin{scope}
    \node [place] (w1)      [label=above:$exec1$] {};
    \node [place] (w2)      [below of=w1, label=above:$exec2$] {};
    \node [place,tokens=1] (s)  [below of=c1] {};

    \node [transition] (e1) [below left of=w2,label=right:$P2$] {}
      edge [pre,bend left]                  (w2)
      edge [post,bend right]                (s);

    \node [transition] (e2) [left of=e1,label=right:$P1$] {}
      edge [pre,bend left]                  (w1)
      edge [post,bend right]                (s);

   \node [transition] (l1) [below right of=w2,label=left:$V2$] {}
      edge [pre,bend left]                  (s)
      edge [post,bend right] node[swap] {} (w2);
      
   \node [transition] (l2) [right of=l1,label=left:$V1$] {}
      edge [pre,bend left]                  (s)
      edge [post,bend right] node[swap] {} (w1);
  \end{scope}
\end{tikzpicture}
\end{center}

Par définition d'un sémaphore, $P$ et $V$ sont totalement
parallèlisés, et ne sont donc pas en concurrence l'un par rapport à
l'autre. Cependant, ils peuvent être en concurrence par rapport à
eux-même.

\subsection*{Question 2}

On représente une couleur par processus.

TODO (below : just an example in tikz)

\begin{tikzpicture}[node distance=1.3cm,>=stealth',bend angle=45,auto]

  \tikzstyle{place}=[circle,thick,draw=blue!75,fill=blue!20,minimum size=6mm]
  \tikzstyle{red place}=[place,draw=red!75,fill=red!20]
  \tikzstyle{transition}=[rectangle,thick,draw=black!75,
  			  fill=black!20,minimum size=4mm]

  \tikzstyle{every label}=[red]
 \begin{scope}[xshift=6cm]
    % Second net
    \node [place,tokens=1]
                      (w1')                                                {};
    \node [place]     (c1') [below of=w1']                                 {};
    \node [red place] (s1') [below of=c1',xshift=-5mm,label=left:$s$]      {};
    \node [red place,tokens=3]
                      (s2') [below of=c1',xshift=5mm,label=right:$\bar s$] {};
    \node [place]     (c2') [below of=s1',xshift=5mm]                      {};
    \node [place,tokens=1]
                      (w2') [below of=c2']                                 {};

    \node [transition] (e1') [left of=c1'] {}
      edge [pre,bend left]                  (w1')
      edge [post]                           (s1')
      edge [pre]                            (s2')
      edge [post]                           (c1');

    \node [transition] (e2') [left of=c2'] {}
      edge [pre,bend right]                 (w2')
      edge [post]                           (s1')
      edge [pre]                            (s2')
      edge [post]                           (c2');

    \node [transition] (l1') [right of=c1'] {}
      edge [pre]                            (c1')
      edge [pre]                            (s1')
      edge [post]                           (s2')
      edge [post,bend right] node[swap] {2} (w1');

    \node [transition] (l2') [right of=c2'] {}
      edge [pre]                            (c2')
      edge [pre]                            (s1')
      edge [post]                           (s2')
      edge [post,bend left]  node {2}       (w2');
  \end{scope}

  \begin{pgfonlayer}{background}
    \filldraw [line width=4mm,join=round,black!10]
      (w1'.north -| l1'.east) rectangle (w2'.south -| e1'.west);
  \end{pgfonlayer}
\end{tikzpicture}

Dans la version colorée du réseau de Pétri du sémaphore, $P$ et $V$
demeurent non concurrents l'un par rapport à l'autre, et, grâce aux
couleurs associées aux processus, ne sont pas également pas
concurrents par rapport à eux même.
