\section{Exercice 1 : L'utilisation des réseaux de Pétri pour le
  contrôle commande}
\subsection{Question 1}
\textit{Cf.} figure $3$ de l'énoncé.
\subsection{Question 2}
Notons $A_d$ et $R_d$ respectivement avancer et reculer le pochoir de
droite.  Notons $A_g$ et $R_g$ respectivement avancer et reculer le
pochoir de gauche.  Quand un pochoir avance, il ne recule pas.  On
commence donc pas le faire avancer ($p_3$ et $p_2$) puis reculer
($p_4$ et $p_5$) On distingue un bloc de parallélisme dans la figure 3
qui signifie que les pochoirs droite et gauche sont actionnés en
parallèle.

Au début d'un tour, en $p_1$, le pochoir ne reçoit aucun signal.  

Puis, en $t_1$, l'opération commence car on a les signaux $p$ et $q$ à
vrai. On rentre ensuite dans le bloc de parallélisme précédemment
décrit, jusqu'à $t_6$ où, après immobilisation à droite et à gauche
des pochoirs (places $p_6$ et $p_7$), un nouveau tour peut commencer
(retour en place $p_1$).
\subsection{Question 3}

\subsection{Question 4}

\subsection{Question 5}

\subsection{Question 6}
Matrice $C$ d'incidence du réseau de pétri de la figure $3$.

 $ \begin{pmatrix}
C&t_1&t_2&t_3&t_4&t_5&t_6 \\
p_1&t_1&t_2&t_3&t_4&t_5&t_6 \\
p_2&t_1&t_2&t_3&t_4&t_5&t_6 \\
p_3&t_1&t_2&t_3&t_4&t_5&t_6 \\
p_4&t_1&t_2&t_3&t_4&t_5&t_6 \\
p_5&t_1&t_2&t_3&t_4&t_5&t_6 \\
p_6&t_1&t_2&t_3&t_4&t_5&t_6 \\
p_7&t_1&t_2&t_3&t_4&t_5&t_6 
\end{pmatrix}$

TODO 
\subsection{Question 7}

\subsection{Question 8}

\subsection{Question 9}
graphe d'évènements ????
