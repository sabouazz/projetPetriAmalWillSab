\section{Exercice 1}
\subsection{Question 1}
à faire sous tina
\subsection{Question 2}
En $p_1$, le pochoir ne reçoit aucun signal.
En $t_1$
Notons $A_d$ et $R_d$ respectivement avancer et reculer le pochoir de
droite.
Notons $A_g$ et $R_g$ respectivement avancer et reculer le pochoir de gauche.
Quand un pochoir avance, il ne recule pas.
On commence donc pas le faire avancer ($p_3$ et $p_2$) puis reculer ($p_4$ et $p_5$)
On distingue un bloc parallélisme dans la figure 3 qui signie que les
pochoirs droite et gauche sont actionnés en parallèle.

\subsection{Question 3}

\subsection{Question 3}

\subsection{Question 4}

\subsection{Question 5}

\subsection{Question 6}

\subsection{Question 7}

\subsection{Question 8}

\subsection{Question 9}
