\section*{Exercice 3~: Étude d'un réseau de Petri particulier.}
\subsection*{Question 1}
TODO
\subsection*{Question 2}
TODO
\subsection*{Question 3}

Arborescence de couverture figure $4$ :

TODO (below : just an example in tikz)

Nous considérons le marquage associé aux places $(c_0, c'_0, d_0, d_1,
d_2, f_0)$.
\begin{center}
\begin{tikzpicture}[->,>=stealth',shorten >=1pt,auto,node distance=2cm,
  thick,main
  node/.style={circle,fill=white,draw,font=\sffamily\Large\bfseries},
  inner sep = 0.5pt]
  \node[main node] (1)[label=center:$\langle n\ 0\ 1\ 0\ 0\ 0 \rangle$] {} ;
  \node[main node] (2) [below of=1] [label=center:$\langle n-1\ 2\ 1\ 1\ 0\ 0 \rangle$] {};
  \node[main node] (3) [below left of=2]  [label=center:$\langle n-2\ 4\ 1\ 1\ 0\ 0 \rangle$] {};
  \node[main node] (4) [below of=2]  [label=center:$\langle n\ 1\ 1\ 1\ 1\ 0 \rangle$] {};
  \node[main node] (5) [below right of=2]  [label=center:$\langle
    n-1\ 2\ 1\ 0\ 1\ 0 \rangle$] {};
 \node[main node] (6) [below right of=2] [label=center:$\langle n\ 0\ 0\ 1\ 0\ 0 \rangle$] {} ;
  \node[main node] (7) [below left of=3]  [label=center:$\langle
    n-2\ 4\ 1\ 1\ 0\ 0 \rangle$] {?};
 \node[main node] (8) [below  of=3]  [label=center:$\langle
   n-2\ 4\ 1\ 1\ 0\ 0 \rangle$] {?};
 \node[main node] (9) [below right of=3]  [label=center:$\langle
   n-2\ 4\ 1\ 1\ 0\ 0 \rangle$] {?};
 \node[main node] (10) [below right of=3]  [label=center:$\langle n-2\ 4\ 0\ 2\ 0\ 0 \rangle$] {};

  \path[every node/.style={font=\sffamily\small}]
    (1)  edge node [right] {$t_1$} (2)
        % edge [bend right] node[left] {0.3} (2)
        % boucles edge [loop above] node {0.1} (1)
    (2)  edge node [left] {$t_1$} (3)
         edge node [right] {$t_2$} (4)
         edge node [right] {$t_4$} (5)
         edge node [right] {$t_3$} (6)
   (3)   edge node [right] {$t_1$} (7)
         edge node [right] {$t_4$} (8)
         edge node [right] {$t_2$} (9)
         edge node [right] {$t_3$} (10);
        
\end{tikzpicture}
\end{center}

Graphe de couverture figure $4$ :
TODO !!

(avec cycles)


Arborescence de couverture figure $5$ :

TODO !!

Nous considérons le marquage associé aux places $(d_i, d_{i-1},
f_{i-1}, f', f_i, c'_i, c_i, c_{i-1})$.

\begin{tikzpicture}[->,>=stealth',shorten >=1pt,auto,node distance=3cm,
  thick,main
  node/.style={circle,fill=white,draw,font=\sffamily\Large\bfseries},
  inner sep = 0pt]
  \node[main node] (1)[label=center:$\langle 1\ 0\ 0\ 0\ 0\ 0 \ 0\ 0\rangle$] {} ;
  \node[main node] (2) [below left of=1] {};
  \node[main node] (3) [below right of=1] {};

  \path[every node/.style={font=\sffamily\small}]
    (1)  edge node [right] {} (2)
         edge node [right] {} (3);

        
\end{tikzpicture}

Graohe  de couverture figure $5$ :

TODO !!




\subsection*{Question 4}
TODO
